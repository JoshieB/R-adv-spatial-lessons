\documentclass[]{book}
\usepackage{lmodern}
\usepackage{amssymb,amsmath}
\usepackage{ifxetex,ifluatex}
\usepackage{fixltx2e} % provides \textsubscript
\ifnum 0\ifxetex 1\fi\ifluatex 1\fi=0 % if pdftex
  \usepackage[T1]{fontenc}
  \usepackage[utf8]{inputenc}
\else % if luatex or xelatex
  \ifxetex
    \usepackage{mathspec}
  \else
    \usepackage{fontspec}
  \fi
  \defaultfontfeatures{Ligatures=TeX,Scale=MatchLowercase}
\fi
% use upquote if available, for straight quotes in verbatim environments
\IfFileExists{upquote.sty}{\usepackage{upquote}}{}
% use microtype if available
\IfFileExists{microtype.sty}{%
\usepackage{microtype}
\UseMicrotypeSet[protrusion]{basicmath} % disable protrusion for tt fonts
}{}
\usepackage[margin=1in]{geometry}
\usepackage{hyperref}
\hypersetup{unicode=true,
            pdftitle={R Advanced Spatial Lessons},
            pdfauthor={Ben Best},
            pdfborder={0 0 0},
            breaklinks=true}
\urlstyle{same}  % don't use monospace font for urls
\usepackage{color}
\usepackage{fancyvrb}
\newcommand{\VerbBar}{|}
\newcommand{\VERB}{\Verb[commandchars=\\\{\}]}
\DefineVerbatimEnvironment{Highlighting}{Verbatim}{commandchars=\\\{\}}
% Add ',fontsize=\small' for more characters per line
\usepackage{framed}
\definecolor{shadecolor}{RGB}{248,248,248}
\newenvironment{Shaded}{\begin{snugshade}}{\end{snugshade}}
\newcommand{\KeywordTok}[1]{\textcolor[rgb]{0.13,0.29,0.53}{\textbf{{#1}}}}
\newcommand{\DataTypeTok}[1]{\textcolor[rgb]{0.13,0.29,0.53}{{#1}}}
\newcommand{\DecValTok}[1]{\textcolor[rgb]{0.00,0.00,0.81}{{#1}}}
\newcommand{\BaseNTok}[1]{\textcolor[rgb]{0.00,0.00,0.81}{{#1}}}
\newcommand{\FloatTok}[1]{\textcolor[rgb]{0.00,0.00,0.81}{{#1}}}
\newcommand{\ConstantTok}[1]{\textcolor[rgb]{0.00,0.00,0.00}{{#1}}}
\newcommand{\CharTok}[1]{\textcolor[rgb]{0.31,0.60,0.02}{{#1}}}
\newcommand{\SpecialCharTok}[1]{\textcolor[rgb]{0.00,0.00,0.00}{{#1}}}
\newcommand{\StringTok}[1]{\textcolor[rgb]{0.31,0.60,0.02}{{#1}}}
\newcommand{\VerbatimStringTok}[1]{\textcolor[rgb]{0.31,0.60,0.02}{{#1}}}
\newcommand{\SpecialStringTok}[1]{\textcolor[rgb]{0.31,0.60,0.02}{{#1}}}
\newcommand{\ImportTok}[1]{{#1}}
\newcommand{\CommentTok}[1]{\textcolor[rgb]{0.56,0.35,0.01}{\textit{{#1}}}}
\newcommand{\DocumentationTok}[1]{\textcolor[rgb]{0.56,0.35,0.01}{\textbf{\textit{{#1}}}}}
\newcommand{\AnnotationTok}[1]{\textcolor[rgb]{0.56,0.35,0.01}{\textbf{\textit{{#1}}}}}
\newcommand{\CommentVarTok}[1]{\textcolor[rgb]{0.56,0.35,0.01}{\textbf{\textit{{#1}}}}}
\newcommand{\OtherTok}[1]{\textcolor[rgb]{0.56,0.35,0.01}{{#1}}}
\newcommand{\FunctionTok}[1]{\textcolor[rgb]{0.00,0.00,0.00}{{#1}}}
\newcommand{\VariableTok}[1]{\textcolor[rgb]{0.00,0.00,0.00}{{#1}}}
\newcommand{\ControlFlowTok}[1]{\textcolor[rgb]{0.13,0.29,0.53}{\textbf{{#1}}}}
\newcommand{\OperatorTok}[1]{\textcolor[rgb]{0.81,0.36,0.00}{\textbf{{#1}}}}
\newcommand{\BuiltInTok}[1]{{#1}}
\newcommand{\ExtensionTok}[1]{{#1}}
\newcommand{\PreprocessorTok}[1]{\textcolor[rgb]{0.56,0.35,0.01}{\textit{{#1}}}}
\newcommand{\AttributeTok}[1]{\textcolor[rgb]{0.77,0.63,0.00}{{#1}}}
\newcommand{\RegionMarkerTok}[1]{{#1}}
\newcommand{\InformationTok}[1]{\textcolor[rgb]{0.56,0.35,0.01}{\textbf{\textit{{#1}}}}}
\newcommand{\WarningTok}[1]{\textcolor[rgb]{0.56,0.35,0.01}{\textbf{\textit{{#1}}}}}
\newcommand{\AlertTok}[1]{\textcolor[rgb]{0.94,0.16,0.16}{{#1}}}
\newcommand{\ErrorTok}[1]{\textcolor[rgb]{0.64,0.00,0.00}{\textbf{{#1}}}}
\newcommand{\NormalTok}[1]{{#1}}
\usepackage{longtable,booktabs}
\usepackage{graphicx,grffile}
\makeatletter
\def\maxwidth{\ifdim\Gin@nat@width>\linewidth\linewidth\else\Gin@nat@width\fi}
\def\maxheight{\ifdim\Gin@nat@height>\textheight\textheight\else\Gin@nat@height\fi}
\makeatother
% Scale images if necessary, so that they will not overflow the page
% margins by default, and it is still possible to overwrite the defaults
% using explicit options in \includegraphics[width, height, ...]{}
\setkeys{Gin}{width=\maxwidth,height=\maxheight,keepaspectratio}
\IfFileExists{parskip.sty}{%
\usepackage{parskip}
}{% else
\setlength{\parindent}{0pt}
\setlength{\parskip}{6pt plus 2pt minus 1pt}
}
\setlength{\emergencystretch}{3em}  % prevent overfull lines
\providecommand{\tightlist}{%
  \setlength{\itemsep}{0pt}\setlength{\parskip}{0pt}}
\setcounter{secnumdepth}{5}
% Redefines (sub)paragraphs to behave more like sections
\ifx\paragraph\undefined\else
\let\oldparagraph\paragraph
\renewcommand{\paragraph}[1]{\oldparagraph{#1}\mbox{}}
\fi
\ifx\subparagraph\undefined\else
\let\oldsubparagraph\subparagraph
\renewcommand{\subparagraph}[1]{\oldsubparagraph{#1}\mbox{}}
\fi

%%% Use protect on footnotes to avoid problems with footnotes in titles
\let\rmarkdownfootnote\footnote%
\def\footnote{\protect\rmarkdownfootnote}

%%% Change title format to be more compact
\usepackage{titling}

% Create subtitle command for use in maketitle
\newcommand{\subtitle}[1]{
  \posttitle{
    \begin{center}\large#1\end{center}
    }
}

\setlength{\droptitle}{-2em}
  \title{R Advanced Spatial Lessons}
  \pretitle{\vspace{\droptitle}\centering\huge}
  \posttitle{\par}
  \author{Ben Best}
  \preauthor{\centering\large\emph}
  \postauthor{\par}
  \predate{\centering\large\emph}
  \postdate{\par}
  \date{2017-09-24}

\usepackage{booktabs}
\usepackage{amsthm}
\makeatletter
\def\thm@space@setup{%
  \thm@preskip=8pt plus 2pt minus 4pt
  \thm@postskip=\thm@preskip
}
\makeatother

\usepackage{amsthm}
\newtheorem{theorem}{Theorem}[chapter]
\newtheorem{lemma}{Lemma}[chapter]
\theoremstyle{definition}
\newtheorem{definition}{Definition}[chapter]
\newtheorem{corollary}{Corollary}[chapter]
\newtheorem{proposition}{Proposition}[chapter]
\theoremstyle{definition}
\newtheorem{example}{Example}[chapter]
\theoremstyle{definition}
\newtheorem{exercise}{Exercise}[chapter]
\theoremstyle{remark}
\newtheorem*{remark}{Remark}
\newtheorem*{solution}{Solution}
\begin{document}
\maketitle

{
\setcounter{tocdepth}{1}
\tableofcontents
}
\chapter*{Prerequisites}\label{prereq}
\addcontentsline{toc}{chapter}{Prerequisites}

Lessons presented here are a continuation of the
\href{http://www.datacarpentry.org/lessons/\#geospatial-data-workshop}{Geospatial
workshop using R of Data Carpentry} described more specifically for the
\href{https://jsta.github.io/2017-09-27-LBNL/}{Lawrence Berkeley
National Lab: Sep 27-28, 2017}.

This content is setup for now using
\href{http://bookdown.org/yihui/bookdown}{bookdown} (using the
\href{https://github.com/rstudio/bookdown-demo}{bookdown-demo}) for
expediency, and meant to eventually be folded into the
\href{https://github.com/swcarpentry/styles}{Software Carpentry style}.

\chapter{Tidy Spatial Analysis}\label{tidy}

\section{Overview}\label{overview}

\textbf{Questions}

\begin{itemize}
\tightlist
\item
  How to elegantly conduct complex spatial analysis by chaining
  operations?
\item
  What is the percent area of water by region across the United States?
\end{itemize}

\textbf{Objectives}

\begin{itemize}
\tightlist
\item
  Use the \texttt{\%\textgreater{}\%} operator (aka ``then'' or
  ``pipe'') to pass output from one function into input of the next.
\item
  Calculate metrics on spatial attributes.
\item
  Aggregate spatial data with metrics.
\item
  Display a map of results.
\end{itemize}

\section{Prerequisites}\label{prerequisites}

\textbf{R Skill Level}: Intermediate - you've got basics of R down.

You will use the \texttt{sf} package for vector data along with the
\texttt{dplyr} package for calculating and manipulating attribute data.

\begin{Shaded}
\begin{Highlighting}[]
\CommentTok{# load packages}
\KeywordTok{library}\NormalTok{(tidyverse)  }\CommentTok{# load dplyr, tidyr, ggplot2 packages}
\KeywordTok{library}\NormalTok{(sf)         }\CommentTok{# vector reading & analysis}

\CommentTok{# set working directory to data folder}
\CommentTok{# setwd("pathToDirHere")}
\end{Highlighting}
\end{Shaded}

\section{States: read and plot}\label{states-read-and-plot}

Similar to
\href{https://jsta.github.io/R-spatial-raster-vector-lesson/09-vector-when-data-dont-line-up-crs/}{Lesson
9: Handling Spatial Projection \& CRS in R}, we'll start by reading in a
polygon shapefile using the \texttt{sf} package. Then use the default
\texttt{plot()} function to see what it looks like.

\begin{Shaded}
\begin{Highlighting}[]
\CommentTok{# read in states}
\NormalTok{states <-}\StringTok{ }\KeywordTok{read_sf}\NormalTok{(}\StringTok{"data/NEON-DS-Site-Layout-Files/US-Boundary-Layers/US-State-Boundaries-Census-2014.shp"}\NormalTok{)}

\CommentTok{# plot the states}
\KeywordTok{plot}\NormalTok{(states)}
\end{Highlighting}
\end{Shaded}

\begin{verbatim}
## Warning: plotting the first 9 out of 10 attributes; use max.plot = 10 to
## plot all
\end{verbatim}

\includegraphics{R-adv-spatial_files/figure-latex/plot-states-1.pdf}

Notice the default plot on \texttt{sf} objects outputs colorized values
of the first 9 of 10 columns. Use the suggestion from the warning to
plot the 10th column.

\begin{Shaded}
\begin{Highlighting}[]
\CommentTok{# plot 10th column}
\KeywordTok{plot}\NormalTok{(states, }\DataTypeTok{max.plot =} \DecValTok{10}\NormalTok{)}
\end{Highlighting}
\end{Shaded}

\includegraphics{R-adv-spatial_files/figure-latex/plot-states-10-1.pdf}

\begin{Shaded}
\begin{Highlighting}[]
\CommentTok{# show columns of the data frame}
\KeywordTok{names}\NormalTok{(states)}
\end{Highlighting}
\end{Shaded}

\begin{verbatim}
##  [1] "STATEFP"  "STATENS"  "AFFGEOID" "GEOID"    "STUSPS"   "NAME"    
##  [7] "LSAD"     "ALAND"    "AWATER"   "region"   "geometry"
\end{verbatim}

\begin{Shaded}
\begin{Highlighting}[]
\CommentTok{# look at table}
\KeywordTok{glimpse}\NormalTok{(states)}
\end{Highlighting}
\end{Shaded}

\begin{verbatim}
## Observations: 58
## Variables: 11
## $ STATEFP  <chr> "06", "11", "12", "13", "16", "17", "19", "21", "22",...
## $ STATENS  <chr> "01779778", "01702382", "00294478", "01705317", "0177...
## $ AFFGEOID <chr> "0400000US06", "0400000US11", "0400000US12", "0400000...
## $ GEOID    <chr> "06", "11", "12", "13", "16", "17", "19", "21", "22",...
## $ STUSPS   <chr> "CA", "DC", "FL", "GA", "ID", "IL", "IA", "KY", "LA",...
## $ NAME     <chr> "California", "District of Columbia", "Florida", "Geo...
## $ LSAD     <chr> "00", "00", "00", "00", "00", "00", "00", "00", "00",...
## $ ALAND    <dbl> 403483823181, 158350578, 138903200855, 148963503399, ...
## $ AWATER   <dbl> 20483271881, 18633500, 31407883551, 4947080103, 23977...
## $ region   <chr> "West", "Northeast", "Southeast", "Southeast", "West"...
## $ geometry <simple_feature> MULTIPOLYGONZ(((-118.593969..., MULTIPOLYG...
\end{verbatim}

\begin{Shaded}
\begin{Highlighting}[]
\CommentTok{# convert to tibble for nicer printing}
\KeywordTok{as_tibble}\NormalTok{(states)}
\end{Highlighting}
\end{Shaded}

\begin{verbatim}
## Simple feature collection with 58 features and 10 fields
## geometry type:  MULTIPOLYGON
## dimension:      XYZ
## bbox:           xmin: -124.7258 ymin: 24.49813 xmax: -66.9499 ymax: 49.38436
## epsg (SRID):    4326
## proj4string:    +proj=longlat +datum=WGS84 +no_defs
## # A tibble: 58 x 11
##    STATEFP  STATENS    AFFGEOID GEOID STUSPS                 NAME  LSAD
##      <chr>    <chr>       <chr> <chr>  <chr>                <chr> <chr>
##  1      06 01779778 0400000US06    06     CA           California    00
##  2      11 01702382 0400000US11    11     DC District of Columbia    00
##  3      12 00294478 0400000US12    12     FL              Florida    00
##  4      13 01705317 0400000US13    13     GA              Georgia    00
##  5      16 01779783 0400000US16    16     ID                Idaho    00
##  6      17 01779784 0400000US17    17     IL             Illinois    00
##  7      19 01779785 0400000US19    19     IA                 Iowa    00
##  8      21 01779786 0400000US21    21     KY             Kentucky    00
##  9      22 01629543 0400000US22    22     LA            Louisiana    00
## 10      24 01714934 0400000US24    24     MD             Maryland    00
## # ... with 48 more rows, and 4 more variables: ALAND <dbl>, AWATER <dbl>,
## #   region <chr>, geometry <simple_feature>
\end{verbatim}

\begin{Shaded}
\begin{Highlighting}[]
\KeywordTok{names}\NormalTok{(states)}
\end{Highlighting}
\end{Shaded}

\begin{verbatim}
##  [1] "STATEFP"  "STATENS"  "AFFGEOID" "GEOID"    "STUSPS"   "NAME"    
##  [7] "LSAD"     "ALAND"    "AWATER"   "region"   "geometry"
\end{verbatim}

\begin{Shaded}
\begin{Highlighting}[]
\CommentTok{# inspect the class(es) of the states object}
\KeywordTok{class}\NormalTok{(states)}
\end{Highlighting}
\end{Shaded}

\begin{verbatim}
## [1] "sf"         "tbl_df"     "tbl"        "data.frame"
\end{verbatim}

The class of the \texttt{states} object is both a simple feature
(\texttt{sf}) as well as a data frame, which means the many useful
functions available to a data frame (or ``tibble'') can be applied.

To plot the column of interest, feed the ``slice'' of that column to the
\texttt{plot()} function.

\begin{Shaded}
\begin{Highlighting}[]
\KeywordTok{plot}\NormalTok{(states[}\StringTok{'region'}\NormalTok{])}
\end{Highlighting}
\end{Shaded}

\includegraphics{R-adv-spatial_files/figure-latex/plot-states-region-1.pdf}

\textbf{Question}: To motivate the spatial analysis for the rest of this
lesson, you will answer this question: ``\emph{\textbf{What is the
percent water by region?}}''

\section{Challenge: analytical steps?}\label{challenge-analytical-steps}

Outline a sequence of analytical steps needed to arrive at the answer.

\subsection{Answers}\label{answers}

\begin{enumerate}
\def\labelenumi{\arabic{enumi}.}
\tightlist
\item
  \textbf{Sum} the area of water (\texttt{AWATER}) and land
  (\texttt{ALAND}) per region.
\item
  \textbf{Divide} the area of water (\texttt{AWATER}) by the area of
  land (\texttt{ALAND}) per region to arrive at percent water.
\item
  Show \textbf{table} of regions sorted by percent water.
\item
  Show \textbf{map} of regions by percent water with a color ramp and
  legend.
\end{enumerate}

\section{Regions: calculate \% water}\label{regions-calculate-water}

\begin{itemize}
\tightlist
\item
  Use the \texttt{\%\textgreater{}\%} operator (aka ``then'' or
  ``pipe'') to pass output from one function into input of the next.

  \begin{itemize}
  \tightlist
  \item
    In RStudio, see menu Help \textgreater{} Keyboard Shortcuts Help for
    a shortcut to the ``Insert Pipe Operator''.
  \end{itemize}
\item
  Calculate metrics on spatial attributes.

  \begin{itemize}
  \tightlist
  \item
    In RStudio, see menu Help \textgreater{} Cheatsheets \textgreater{}
    \href{https://github.com/rstudio/cheatsheets/raw/master/source/pdfs/data-transformation-cheatsheet.pdf}{Data
    Manipulation with dplyr, tidyr}.
  \end{itemize}
\item
  Aggregate spatial data with metrics.
\end{itemize}

\begin{Shaded}
\begin{Highlighting}[]
\NormalTok{regions =}\StringTok{ }\NormalTok{states %>%}
\StringTok{  }\KeywordTok{group_by}\NormalTok{(region) %>%}
\StringTok{  }\KeywordTok{summarize}\NormalTok{(}
    \DataTypeTok{water =} \KeywordTok{sum}\NormalTok{(AWATER),}
    \DataTypeTok{land  =} \KeywordTok{sum}\NormalTok{(ALAND)) %>%}
\StringTok{  }\KeywordTok{mutate}\NormalTok{(}
    \DataTypeTok{pct_water =} \NormalTok{water /}\StringTok{ }\NormalTok{land *}\StringTok{ }\DecValTok{100} \NormalTok\StringTok{ }\KeywordTok{round}\NormalTok{(}\DecValTok{2}\NormalTok{))}

\CommentTok{# object}
\NormalTok{regions}
\end{Highlighting}
\end{Shaded}

\begin{verbatim}
## Simple feature collection with 5 features and 4 fields
## geometry type:  GEOMETRY
## dimension:      XYZ
## bbox:           xmin: -124.7258 ymin: 24.49813 xmax: -66.9499 ymax: 49.38436
## epsg (SRID):    4326
## proj4string:    +proj=longlat +datum=WGS84 +no_defs
## # A tibble: 5 x 5
##      region        water         land pct_water          geometry
##       <chr>        <dbl>        <dbl>     <dbl>  <simple_feature>
## 1   Midwest 184383393833 1.943869e+12  9.485380 <MULTIPOLYGON...>
## 2 Northeast 108922434345 8.690661e+11 12.533273 <MULTIPOLYGON...>
## 3 Southeast 103876652998 1.364632e+12  7.612063 <MULTIPOLYGON...>
## 4 Southwest  24217682268 1.462632e+12  1.655761 <POLYGONZ((-9...>
## 5      West  57568049509 2.432336e+12  2.366780 <MULTIPOLYGON...>
\end{verbatim}

Notice the geometry in the column. To remove the geometry column pipe to
\texttt{st\_set\_geometry(NULL)}. To arrange in descending order use
\texttt{arrange(desc(pct\_water))}.

\begin{Shaded}
\begin{Highlighting}[]
\CommentTok{# table}
\NormalTok{regions %>%}
\StringTok{  }\KeywordTok{st_set_geometry}\NormalTok{(}\OtherTok{NULL}\NormalTok{) %>%}
\StringTok{  }\KeywordTok{arrange}\NormalTok{(}\KeywordTok{desc}\NormalTok{(pct_water))}
\end{Highlighting}
\end{Shaded}

\begin{verbatim}
## # A tibble: 5 x 4
##      region        water         land pct_water
##       <chr>        <dbl>        <dbl>     <dbl>
## 1 Northeast 108922434345 8.690661e+11 12.533273
## 2   Midwest 184383393833 1.943869e+12  9.485380
## 3 Southeast 103876652998 1.364632e+12  7.612063
## 4      West  57568049509 2.432336e+12  2.366780
## 5 Southwest  24217682268 1.462632e+12  1.655761
\end{verbatim}

\section{Regions: plot}\label{regions-plot}

Now plot the regions.

\begin{Shaded}
\begin{Highlighting}[]
\CommentTok{# plot, default}
\KeywordTok{plot}\NormalTok{(regions[}\StringTok{'pct_water'}\NormalTok{])}
\end{Highlighting}
\end{Shaded}

\includegraphics{R-adv-spatial_files/figure-latex/plot-regions-pctwater-1.pdf}

\section{Regions: ggplot}\label{regions-ggplot}

The \texttt{ggplot2} library can
\href{http://ggplot2.tidyverse.org/reference/ggsf.html}{visualise sf
objects}.

\begin{itemize}
\tightlist
\item
  In RStudio, see menu Help \textgreater{} Cheatsheets \textgreater{}
  \href{https://github.com/rstudio/cheatsheets/raw/master/source/pdfs/ggplot2-cheatsheet-2.1.pdf}{Data
  Visualization with ggplot2}.
\end{itemize}

\begin{Shaded}
\begin{Highlighting}[]
\CommentTok{# plot, ggplot}
\KeywordTok{ggplot}\NormalTok{(regions) +}
\StringTok{  }\KeywordTok{geom_sf}\NormalTok{(}\KeywordTok{aes}\NormalTok{(}\DataTypeTok{fill =} \NormalTok{pct_water)) +}
\StringTok{  }\KeywordTok{scale_fill_distiller}\NormalTok{(}
    \StringTok{"pct_water"}\NormalTok{, }\DataTypeTok{palette =} \StringTok{"Spectral"}\NormalTok{, }\DataTypeTok{direction=}\DecValTok{1}\NormalTok{,}
    \DataTypeTok{guide =} \KeywordTok{guide_legend}\NormalTok{(}\DataTypeTok{title =} \StringTok{"% water"}\NormalTok{, }\DataTypeTok{reverse=}\NormalTok{T)) +}
\StringTok{  }\KeywordTok{theme_bw}\NormalTok{() +}
\StringTok{  }\KeywordTok{ggtitle}\NormalTok{(}\StringTok{"% Water by US Region"}\NormalTok{)}
\end{Highlighting}
\end{Shaded}

\includegraphics{R-adv-spatial_files/figure-latex/ggplot-regions-pctwater-1.pdf}

\section{Regions: recalculate area}\label{regions-recalculate-area}

So far you've used the \texttt{ALAND} column for area of the state. But
what if you were not provided the area and needed to calculate it?
Because the \texttt{states} are in geographic coordinates, you'll need
to either transform to an equal area projection and calculate area, or
use geodesic calculations. Thankfully, the \texttt{sf} library provides
area calculations with the \texttt{st\_area()} and uses the
\texttt{geosphere::distGeo()} to perform geodesic calculations (ie
trigonometric calculation accounting for the spheroid nature of the
earth). Since the \texttt{states} data has the unusual aspect of a z
dimension, you'll need to first remove that with the \texttt{st\_zm()}
function.

\begin{Shaded}
\begin{Highlighting}[]
\KeywordTok{library}\NormalTok{(geosphere)}
\KeywordTok{library}\NormalTok{(units)}

\NormalTok{regions =}\StringTok{ }\NormalTok{states %>%}
\StringTok{  }\KeywordTok{mutate}\NormalTok{(}
    \DataTypeTok{water_m2 =} \NormalTok{AWATER %>%}\StringTok{ }\KeywordTok{set_units}\NormalTok{(m^}\DecValTok{2}\NormalTok{),}
    \DataTypeTok{land_m2  =} \NormalTok{geometry %>%}\StringTok{ }\KeywordTok{st_zm}\NormalTok{() %>%}\StringTok{ }\KeywordTok{st_area}\NormalTok{()) %>%}
\StringTok{  }\KeywordTok{group_by}\NormalTok{(region) %>%}
\StringTok{  }\KeywordTok{summarize}\NormalTok{(}
    \DataTypeTok{water_m2 =} \KeywordTok{sum}\NormalTok{(water_m2),}
    \DataTypeTok{land_m2  =} \KeywordTok{sum}\NormalTok{(land_m2)) %>%}
\StringTok{  }\KeywordTok{mutate}\NormalTok{(}
    \DataTypeTok{pct_water =} \NormalTok{water_m2 /}\StringTok{ }\NormalTok{land_m2)}

\CommentTok{# table}
\NormalTok{regions %>%}
\StringTok{  }\KeywordTok{st_set_geometry}\NormalTok{(}\OtherTok{NULL}\NormalTok{) %>%}
\StringTok{  }\KeywordTok{arrange}\NormalTok{(}\KeywordTok{desc}\NormalTok{(pct_water))}
\end{Highlighting}
\end{Shaded}

\begin{verbatim}
## # A tibble: 5 x 4
##      region         water_m2          land_m2    pct_water
##       <chr>          <units>          <units>      <units>
## 1 Northeast 108922434345 m^2 9.117041e+11 m^2 0.11947126 1
## 2   Midwest 184383393833 m^2 1.987268e+12 m^2 0.09278233 1
## 3 Southeast 103876652998 m^2 1.427079e+12 m^2 0.07278971 1
## 4      West  57568049509 m^2 2.467170e+12 m^2 0.02333363 1
## 5 Southwest  24217682268 m^2 1.483765e+12 m^2 0.01632178 1
\end{verbatim}

\begin{Shaded}
\begin{Highlighting}[]
\CommentTok{# plot, ggplot}
\KeywordTok{ggplot}\NormalTok{(regions) +}
\StringTok{  }\KeywordTok{geom_sf}\NormalTok{(}\KeywordTok{aes}\NormalTok{(}\DataTypeTok{fill =} \KeywordTok{as.numeric}\NormalTok{(pct_water))) +}
\StringTok{  }\KeywordTok{scale_fill_distiller}\NormalTok{(}
    \StringTok{"pct_water"}\NormalTok{, }\DataTypeTok{palette =} \StringTok{"Spectral"}\NormalTok{, }\DataTypeTok{direction=}\DecValTok{1}\NormalTok{,}
    \DataTypeTok{guide =} \KeywordTok{guide_legend}\NormalTok{(}\DataTypeTok{title =} \StringTok{"% water"}\NormalTok{, }\DataTypeTok{reverse=}\NormalTok{T)) +}
\StringTok{  }\KeywordTok{theme_bw}\NormalTok{() +}
\StringTok{  }\KeywordTok{ggtitle}\NormalTok{(}\StringTok{"% Water by US Region"}\NormalTok{)}
\end{Highlighting}
\end{Shaded}

\includegraphics{R-adv-spatial_files/figure-latex/plot-regions-area-1.pdf}

\section{Challenge: project \& recalculate
area}\label{challenge-project-recalculate-area}

Use \texttt{st\_transform()} with a
\href{http://spatialreference.org/ref/esri/usa-contiguous-albers-equal-area-conic/}{USA
Contiguous Albers Equal Area Conic Projection} that minimizes
distoration, and then calculate area using the \texttt{st\_area()}
function.

\subsection{Answers}\label{answers-1}

\begin{Shaded}
\begin{Highlighting}[]
\KeywordTok{library}\NormalTok{(geosphere)}
\KeywordTok{library}\NormalTok{(units)}

\CommentTok{# Proj4 of http://spatialreference.org/ref/esri/usa-contiguous-albers-equal-area-conic/}
\NormalTok{crs_usa =}\StringTok{ '+proj=aea +lat_1=29.5 +lat_2=45.5 +lat_0=37.5 +lon_0=-96 +x_0=0 +y_0=0 +ellps=GRS80 +datum=NAD83 +units=m +no_defs'}

\NormalTok{regions =}\StringTok{ }\NormalTok{states %>%}
\StringTok{  }\KeywordTok{st_transform}\NormalTok{(crs_usa) %>%}
\StringTok{  }\KeywordTok{mutate}\NormalTok{(}
    \DataTypeTok{water_m2 =} \NormalTok{AWATER %>%}\StringTok{ }\KeywordTok{set_units}\NormalTok{(m^}\DecValTok{2}\NormalTok{),}
    \DataTypeTok{land_m2  =} \NormalTok{geometry %>%}\StringTok{ }\KeywordTok{st_zm}\NormalTok{() %>%}\StringTok{ }\KeywordTok{st_area}\NormalTok{()) %>%}
\StringTok{  }\KeywordTok{group_by}\NormalTok{(region) %>%}
\StringTok{  }\KeywordTok{summarize}\NormalTok{(}
    \DataTypeTok{water_m2 =} \KeywordTok{sum}\NormalTok{(water_m2),}
    \DataTypeTok{land_m2  =} \KeywordTok{sum}\NormalTok{(land_m2)) %>%}
\StringTok{  }\KeywordTok{mutate}\NormalTok{(}
    \DataTypeTok{pct_water =} \NormalTok{water_m2 /}\StringTok{ }\NormalTok{land_m2)}

\CommentTok{# table}
\NormalTok{regions %>%}
\StringTok{  }\KeywordTok{st_set_geometry}\NormalTok{(}\OtherTok{NULL}\NormalTok{) %>%}
\StringTok{  }\KeywordTok{arrange}\NormalTok{(}\KeywordTok{desc}\NormalTok{(pct_water))}
\end{Highlighting}
\end{Shaded}

\begin{verbatim}
## # A tibble: 5 x 4
##      region         water_m2          land_m2    pct_water
##       <chr>          <units>          <units>      <units>
## 1 Northeast 108922434345 m^2 9.117031e+11 m^2 0.11947138 1
## 2   Midwest 184383393833 m^2 1.987266e+12 m^2 0.09278246 1
## 3 Southeast 103876652998 m^2 1.427078e+12 m^2 0.07278973 1
## 4      West  57568049509 m^2 2.467167e+12 m^2 0.02333367 1
## 5 Southwest  24217682268 m^2 1.483758e+12 m^2 0.01632185 1
\end{verbatim}

\begin{Shaded}
\begin{Highlighting}[]
\CommentTok{# plot, ggplot}
\KeywordTok{ggplot}\NormalTok{(regions) +}
\StringTok{  }\KeywordTok{geom_sf}\NormalTok{(}\KeywordTok{aes}\NormalTok{(}\DataTypeTok{fill =} \KeywordTok{as.numeric}\NormalTok{(pct_water))) +}
\StringTok{  }\KeywordTok{scale_fill_distiller}\NormalTok{(}\StringTok{"pct_water"}\NormalTok{, }\DataTypeTok{palette =} \StringTok{"Spectral"}\NormalTok{) +}
\StringTok{  }\KeywordTok{theme_bw}\NormalTok{() +}
\StringTok{  }\KeywordTok{ggtitle}\NormalTok{(}\StringTok{"% Water (geodesic) by US Region"}\NormalTok{)}
\end{Highlighting}
\end{Shaded}

\includegraphics{R-adv-spatial_files/figure-latex/recalc-regions-transform-1.pdf}

\section{Key Points}\label{key-points}

\begin{itemize}
\tightlist
\item
  The \texttt{sf} package can take advantage of chaining spatial
  operations using the \texttt{\%\textgreater{}\%} operator.
\item
  Data manipulation functions in \texttt{dplyr} such as
  \texttt{group\_by()}, \texttt{summarize()} and \texttt{mutate()} work
  on \texttt{sf} objects.
\item
  Area can be calculated a variety of ways. Geodesic is preferred if
  starting with geographic coordinates (vs projected).
\end{itemize}

\chapter{Interactive Maps}\label{interactive}

\section{Overview}\label{overview-1}

\textbf{Questions}

\begin{itemize}
\tightlist
\item
  How do you generate interactive plots of spatial data to enable pan,
  zoom and hover/click for more detail?
\end{itemize}

\textbf{Objectives}

Learn variety of methods for producing interactive spatial output using
libraries:

\begin{itemize}
\tightlist
\item
  \texttt{plotly}: makes any ggplot2 object interactive
\item
  \texttt{mapview}: quick view of any spatial object
\item
  \texttt{leaflet}: full control over interactive map
\end{itemize}

\section{Things You'll Need to Complete this
Tutorial}\label{things-youll-need-to-complete-this-tutorial}

\textbf{R Skill Level}: Intermediate - you've got basics of R down.

We will continue to use the \texttt{sf} and \texttt{raster} packages and
introduce the \texttt{plotly}, \texttt{mapview}, and \texttt{leaflet}
packages in this tutorial.

\begin{Shaded}
\begin{Highlighting}[]
\CommentTok{# load packages}
\KeywordTok{library}\NormalTok{(tidyverse)  }\CommentTok{# loads dplyr, tidyr, ggplot2 packages}
\KeywordTok{library}\NormalTok{(sf)         }\CommentTok{# simple features package - vector}
\KeywordTok{library}\NormalTok{(raster)     }\CommentTok{# raster}
\KeywordTok{library}\NormalTok{(plotly)     }\CommentTok{# makes ggplot objects interactive}
\KeywordTok{library}\NormalTok{(mapview)    }\CommentTok{# quick interactive viewing of spatial objects}
\KeywordTok{library}\NormalTok{(leaflet)    }\CommentTok{# interactive maps}

\CommentTok{# set working directory to data folder}
\CommentTok{# setwd("pathToDirHere")}
\end{Highlighting}
\end{Shaded}

\section{States: ggplot2}\label{states-ggplot2}

Recreate the ggplot object from Lesson \ref{tidy} and save into a
variable for subsequent use with the \texttt{plotly} package.

\begin{Shaded}
\begin{Highlighting}[]
\CommentTok{# read in states}
\NormalTok{states <-}\StringTok{ }\KeywordTok{read_sf}\NormalTok{(}\StringTok{"data/NEON-DS-Site-Layout-Files/US-Boundary-Layers/US-State-Boundaries-Census-2014.shp"}\NormalTok{) %>%}
\StringTok{  }\KeywordTok{st_zm}\NormalTok{() %>%}
\StringTok{  }\KeywordTok{mutate}\NormalTok{(}
    \DataTypeTok{water_km2 =} \NormalTok{(AWATER /}\StringTok{ }\NormalTok{(}\DecValTok{1000}\NormalTok{*}\DecValTok{1000}\NormalTok{)) %>%}\StringTok{ }\KeywordTok{round}\NormalTok{(}\DecValTok{2}\NormalTok{))}

\CommentTok{# plot, ggplot}
\NormalTok{g =}\StringTok{ }\KeywordTok{ggplot}\NormalTok{(states) +}
\StringTok{  }\KeywordTok{geom_sf}\NormalTok{(}\KeywordTok{aes}\NormalTok{(}\DataTypeTok{fill =} \NormalTok{water_km2)) +}
\StringTok{  }\KeywordTok{scale_fill_distiller}\NormalTok{(}\StringTok{"water_km2"}\NormalTok{, }\DataTypeTok{palette =} \StringTok{"Spectral"}\NormalTok{) +}
\StringTok{  }\KeywordTok{ggtitle}\NormalTok{(}\StringTok{"Water (km2) by State"}\NormalTok{)}
\NormalTok{g}
\end{Highlighting}
\end{Shaded}

\includegraphics{R-adv-spatial_files/figure-latex/states-ggplot2-1.pdf}

\section{States: plotly}\label{states-plotly}

The \texttt{plotly::ggplotly()} function outputs a ggplot into an
interactive window capable of pan, zoom and identify.

\begin{Shaded}
\begin{Highlighting}[]
\KeywordTok{library}\NormalTok{(plotly)}

\KeywordTok{ggplotly}\NormalTok{(g)}
\end{Highlighting}
\end{Shaded}

\includegraphics{R-adv-spatial_files/figure-latex/states-plotly-1.pdf}

\section{States: mapview}\label{states-mapview}

The \texttt{mapview::mapview()} function can work for a quick view of
the data, providing choropleths, background maps and attribute popups.
Performance varies on the object and customization can be tricky.

\begin{Shaded}
\begin{Highlighting}[]
\KeywordTok{library}\NormalTok{(mapview)}

\CommentTok{# simple view with popups}
\KeywordTok{mapview}\NormalTok{(states)}
\end{Highlighting}
\end{Shaded}

\includegraphics{R-adv-spatial_files/figure-latex/states-mapview-1.pdf}

\begin{Shaded}
\begin{Highlighting}[]
\CommentTok{# coloring and layering}
\KeywordTok{mapview}\NormalTok{(states, }\DataTypeTok{zcol=}\StringTok{'water_km2'}\NormalTok{, }\DataTypeTok{burst=}\StringTok{'STUSPS'}\NormalTok{)}
\end{Highlighting}
\end{Shaded}

\includegraphics{R-adv-spatial_files/figure-latex/states-mapview-2.pdf}

\section{States: leaflet}\label{states-leaflet}

The \href{http://rstudio.github.io/leaflet/}{\texttt{leaflet}} package
offers a robust set of functions for viewing vector and raster data,
although requires more explicit functions.

\begin{Shaded}
\begin{Highlighting}[]
\KeywordTok{library}\NormalTok{(leaflet)}

\KeywordTok{leaflet}\NormalTok{(states) %>%}
\StringTok{  }\KeywordTok{addTiles}\NormalTok{() %>%}
\StringTok{  }\KeywordTok{addPolygons}\NormalTok{()}
\end{Highlighting}
\end{Shaded}

\includegraphics{R-adv-spatial_files/figure-latex/states-leaflet-1.pdf}

\subsection{Choropleth}\label{choropleth}

Drawing from the documentation from
\href{http://rstudio.github.io/leaflet/choropleths.html}{Leaflet for R -
Choropleths}, we can construct a pretty choropleth.

\begin{Shaded}
\begin{Highlighting}[]
\NormalTok{pal <-}\StringTok{ }\KeywordTok{colorBin}\NormalTok{(}\StringTok{"Blues"}\NormalTok{, }\DataTypeTok{domain =} \NormalTok{states$water_km2, }\DataTypeTok{bins =} \DecValTok{7}\NormalTok{)}


\KeywordTok{leaflet}\NormalTok{(states) %>%}
\StringTok{  }\KeywordTok{addProviderTiles}\NormalTok{(}\StringTok{"Stamen.TonerLite"}\NormalTok{) %>%}
\StringTok{  }\KeywordTok{addPolygons}\NormalTok{(}
    \CommentTok{# fill}
    \DataTypeTok{fillColor   =} \NormalTok{~}\KeywordTok{pal}\NormalTok{(water_km2),}
    \DataTypeTok{fillOpacity =} \FloatTok{0.7}\NormalTok{,}
    \CommentTok{# line}
    \DataTypeTok{dashArray   =} \StringTok{"3"}\NormalTok{,}
    \DataTypeTok{weight      =} \DecValTok{2}\NormalTok{,}
    \DataTypeTok{color       =} \StringTok{"white"}\NormalTok{,}
    \DataTypeTok{opacity     =} \DecValTok{1}\NormalTok{,}
    \CommentTok{# interaction}
    \DataTypeTok{highlight =} \KeywordTok{highlightOptions}\NormalTok{(}
      \DataTypeTok{weight =} \DecValTok{5}\NormalTok{,}
      \DataTypeTok{color =} \StringTok{"#666"}\NormalTok{,}
      \DataTypeTok{dashArray =} \StringTok{""}\NormalTok{,}
      \DataTypeTok{fillOpacity =} \FloatTok{0.7}\NormalTok{,}
      \DataTypeTok{bringToFront =} \OtherTok{TRUE}\NormalTok{))}
\end{Highlighting}
\end{Shaded}

\includegraphics{R-adv-spatial_files/figure-latex/states-choropleth-1.pdf}

\subsection{Popups and Legend}\label{popups-and-legend}

Adding a legend and popups requires a bit more work, but achieves a very
aesthetically and functionally pleasing visualization.

\begin{Shaded}
\begin{Highlighting}[]
\KeywordTok{library}\NormalTok{(htmltools)}
\KeywordTok{library}\NormalTok{(scales)}

\NormalTok{labels <-}\StringTok{ }\KeywordTok{sprintf}\NormalTok{(}
  \StringTok{"<strong>%s</strong><br/> water: %s km<sup>2</sup>"}\NormalTok{,}
  \NormalTok{states$NAME, }\KeywordTok{comma}\NormalTok{(states$water_km2)) %>%}\StringTok{ }
\StringTok{  }\KeywordTok{lapply}\NormalTok{(HTML)}

\KeywordTok{leaflet}\NormalTok{(states) %>%}
\StringTok{  }\KeywordTok{addProviderTiles}\NormalTok{(}\StringTok{"Stamen.TonerLite"}\NormalTok{) %>%}
\StringTok{  }\KeywordTok{addPolygons}\NormalTok{(}
    \CommentTok{# fill}
    \DataTypeTok{fillColor   =} \NormalTok{~}\KeywordTok{pal}\NormalTok{(water_km2),}
    \DataTypeTok{fillOpacity =} \FloatTok{0.7}\NormalTok{,}
    \CommentTok{# line}
    \DataTypeTok{dashArray   =} \StringTok{"3"}\NormalTok{,}
    \DataTypeTok{weight      =} \DecValTok{2}\NormalTok{,}
    \DataTypeTok{color       =} \StringTok{"white"}\NormalTok{,}
    \DataTypeTok{opacity     =} \DecValTok{1}\NormalTok{,}
    \CommentTok{# interaction}
    \DataTypeTok{highlight =} \KeywordTok{highlightOptions}\NormalTok{(}
      \DataTypeTok{weight =} \DecValTok{5}\NormalTok{,}
      \DataTypeTok{color =} \StringTok{"#666"}\NormalTok{,}
      \DataTypeTok{dashArray =} \StringTok{""}\NormalTok{,}
      \DataTypeTok{fillOpacity =} \FloatTok{0.7}\NormalTok{,}
      \DataTypeTok{bringToFront =} \OtherTok{TRUE}\NormalTok{),}
  \DataTypeTok{label =} \NormalTok{labels,}
  \DataTypeTok{labelOptions =} \KeywordTok{labelOptions}\NormalTok{(}
    \DataTypeTok{style =} \KeywordTok{list}\NormalTok{(}\StringTok{"font-weight"} \NormalTok{=}\StringTok{ "normal"}\NormalTok{, }\DataTypeTok{padding =} \StringTok{"3px 8px"}\NormalTok{),}
    \DataTypeTok{textsize =} \StringTok{"15px"}\NormalTok{,}
    \DataTypeTok{direction =} \StringTok{"auto"}\NormalTok{)) %>%}
\StringTok{  }\KeywordTok{addLegend}\NormalTok{(}
    \DataTypeTok{pal =} \NormalTok{pal, }\DataTypeTok{values =} \NormalTok{~water_km2, }\DataTypeTok{opacity =} \FloatTok{0.7}\NormalTok{, }\DataTypeTok{title =} \KeywordTok{HTML}\NormalTok{(}\StringTok{"Water (km<sup>2</sup>)"}\NormalTok{),}
    \DataTypeTok{position =} \StringTok{"bottomright"}\NormalTok{)}
\end{Highlighting}
\end{Shaded}

\includegraphics{R-adv-spatial_files/figure-latex/states-popups-legend-1.pdf}

\section{Challenge: leaflet for
regions}\label{challenge-leaflet-for-regions}

Use Lesson \ref{tidy} final output to create a regional choropleth with
legend and popups for percent water by region.

\subsection{Answers}\label{answers-2}

\begin{Shaded}
\begin{Highlighting}[]
\NormalTok{regions =}\StringTok{ }\NormalTok{states %>%}
\StringTok{  }\KeywordTok{group_by}\NormalTok{(region) %>%}
\StringTok{  }\KeywordTok{summarize}\NormalTok{(}
    \DataTypeTok{water =} \KeywordTok{sum}\NormalTok{(AWATER),}
    \DataTypeTok{land  =} \KeywordTok{sum}\NormalTok{(ALAND)) %>%}
\StringTok{  }\KeywordTok{mutate}\NormalTok{(}
    \DataTypeTok{pct_water =} \NormalTok{(water /}\StringTok{ }\NormalTok{land *}\StringTok{ }\DecValTok{100}\NormalTok{) %>%}\StringTok{ }\KeywordTok{round}\NormalTok{(}\DecValTok{2}\NormalTok{))}

\NormalTok{pal <-}\StringTok{ }\KeywordTok{colorBin}\NormalTok{(}\StringTok{"Spectral"}\NormalTok{, }\DataTypeTok{domain =} \NormalTok{regions$pct_water, }\DataTypeTok{bins =} \DecValTok{5}\NormalTok{)}

\NormalTok{labels <-}\StringTok{ }\KeywordTok{sprintf}\NormalTok{(}
  \StringTok{"<strong>%s</strong><br/>water: %s%%"}\NormalTok{,}
  \NormalTok{regions$region, }\KeywordTok{comma}\NormalTok{(regions$pct_water)) %>%}\StringTok{ }
\StringTok{  }\KeywordTok{lapply}\NormalTok{(HTML)}

\KeywordTok{leaflet}\NormalTok{(regions) %>%}
\StringTok{  }\KeywordTok{addProviderTiles}\NormalTok{(}\StringTok{"Stamen.TonerLite"}\NormalTok{) %>%}
\StringTok{  }\KeywordTok{addPolygons}\NormalTok{(}
    \CommentTok{# fill}
    \DataTypeTok{fillColor   =} \NormalTok{~}\KeywordTok{pal}\NormalTok{(pct_water),}
    \DataTypeTok{fillOpacity =} \FloatTok{0.7}\NormalTok{,}
    \CommentTok{# line}
    \DataTypeTok{dashArray   =} \StringTok{"3"}\NormalTok{,}
    \DataTypeTok{weight      =} \DecValTok{2}\NormalTok{,}
    \DataTypeTok{color       =} \StringTok{"white"}\NormalTok{,}
    \DataTypeTok{opacity     =} \DecValTok{1}\NormalTok{,}
    \CommentTok{# interaction}
    \DataTypeTok{highlight =} \KeywordTok{highlightOptions}\NormalTok{(}
      \DataTypeTok{weight =} \DecValTok{5}\NormalTok{,}
      \DataTypeTok{color =} \StringTok{"#666"}\NormalTok{,}
      \DataTypeTok{dashArray =} \StringTok{""}\NormalTok{,}
      \DataTypeTok{fillOpacity =} \FloatTok{0.7}\NormalTok{,}
      \DataTypeTok{bringToFront =} \OtherTok{TRUE}\NormalTok{),}
  \DataTypeTok{label =} \NormalTok{labels,}
  \DataTypeTok{labelOptions =} \KeywordTok{labelOptions}\NormalTok{(}
    \DataTypeTok{style =} \KeywordTok{list}\NormalTok{(}\StringTok{"font-weight"} \NormalTok{=}\StringTok{ "normal"}\NormalTok{, }\DataTypeTok{padding =} \StringTok{"3px 8px"}\NormalTok{),}
    \DataTypeTok{textsize =} \StringTok{"15px"}\NormalTok{,}
    \DataTypeTok{direction =} \StringTok{"auto"}\NormalTok{)) %>%}
\StringTok{  }\KeywordTok{addLegend}\NormalTok{(}
    \DataTypeTok{pal =} \NormalTok{pal, }\DataTypeTok{values =} \NormalTok{~pct_water, }\DataTypeTok{opacity =} \FloatTok{0.7}\NormalTok{, }\DataTypeTok{title =} \StringTok{"water %"}\NormalTok{,}
    \DataTypeTok{position =} \StringTok{"bottomright"}\NormalTok{)}
\end{Highlighting}
\end{Shaded}

\includegraphics{R-adv-spatial_files/figure-latex/regions-choropleth-1.pdf}

\section{Raster: leaflet}\label{raster-leaflet}

TODO: show raster overlay using NEON raster dataset example

\section{Key Points}\label{key-points-1}

\begin{itemize}
\tightlist
\item
  Interactive maps provide more detail for visual investigation,
  including use of background maps, but is only relevant in a web
  context.
\item
  Several packages exist for providing interactive views of data.
\item
  The \texttt{plotly::ggplotly()} function works quickly if you already
  have a ggplot object, which is best for static output.
\item
  The \texttt{mapview::mapview()} function can work for a quick view of
  the data, providing choropleths, background maps and attribute popups.
  Performance varies on the object and customization can be tricky.
\item
  The \texttt{leaflet} package provides a highly customizable set of
  functions for rendering of interactive choropleths with background
  maps, legends, etc.
\end{itemize}

\textbf{Tidy Spatial Analysis}

\begin{itemize}
\tightlist
\item
  \href{http://strimas.com/r/tidy-sf/}{Tidy spatial data in R: using
  dplyr, tidyr, and ggplot2 with sf}
\end{itemize}

\textbf{Interactive Maps}

\begin{itemize}
\tightlist
\item
  \href{http://remi-daigle.github.io/2016-04-15-UCSB/viz/}{Visualization
  in R - 2016-04-15-UCSB workshop}
\item
  \href{http://rstudio.github.io/leaflet/}{\texttt{leaflet}}
\item
  \href{http://r-spatial.org/r/2017/01/30/mapedit_intro.html}{\texttt{mapedit}}
\item
  \href{https://r-spatial.github.io/mapview/}{\texttt{mapview}}
\end{itemize}


\end{document}
