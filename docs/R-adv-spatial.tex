\documentclass[]{book}
\usepackage{lmodern}
\usepackage{amssymb,amsmath}
\usepackage{ifxetex,ifluatex}
\usepackage{fixltx2e} % provides \textsubscript
\ifnum 0\ifxetex 1\fi\ifluatex 1\fi=0 % if pdftex
  \usepackage[T1]{fontenc}
  \usepackage[utf8]{inputenc}
\else % if luatex or xelatex
  \ifxetex
    \usepackage{mathspec}
  \else
    \usepackage{fontspec}
  \fi
  \defaultfontfeatures{Ligatures=TeX,Scale=MatchLowercase}
\fi
% use upquote if available, for straight quotes in verbatim environments
\IfFileExists{upquote.sty}{\usepackage{upquote}}{}
% use microtype if available
\IfFileExists{microtype.sty}{%
\usepackage{microtype}
\UseMicrotypeSet[protrusion]{basicmath} % disable protrusion for tt fonts
}{}
\usepackage[margin=1in]{geometry}
\usepackage{hyperref}
\hypersetup{unicode=true,
            pdftitle={R Advanced Spatial Lessons},
            pdfauthor={Ben Best},
            pdfborder={0 0 0},
            breaklinks=true}
\urlstyle{same}  % don't use monospace font for urls
\usepackage{longtable,booktabs}
\usepackage{graphicx,grffile}
\makeatletter
\def\maxwidth{\ifdim\Gin@nat@width>\linewidth\linewidth\else\Gin@nat@width\fi}
\def\maxheight{\ifdim\Gin@nat@height>\textheight\textheight\else\Gin@nat@height\fi}
\makeatother
% Scale images if necessary, so that they will not overflow the page
% margins by default, and it is still possible to overwrite the defaults
% using explicit options in \includegraphics[width, height, ...]{}
\setkeys{Gin}{width=\maxwidth,height=\maxheight,keepaspectratio}
\IfFileExists{parskip.sty}{%
\usepackage{parskip}
}{% else
\setlength{\parindent}{0pt}
\setlength{\parskip}{6pt plus 2pt minus 1pt}
}
\setlength{\emergencystretch}{3em}  % prevent overfull lines
\providecommand{\tightlist}{%
  \setlength{\itemsep}{0pt}\setlength{\parskip}{0pt}}
\setcounter{secnumdepth}{5}
% Redefines (sub)paragraphs to behave more like sections
\ifx\paragraph\undefined\else
\let\oldparagraph\paragraph
\renewcommand{\paragraph}[1]{\oldparagraph{#1}\mbox{}}
\fi
\ifx\subparagraph\undefined\else
\let\oldsubparagraph\subparagraph
\renewcommand{\subparagraph}[1]{\oldsubparagraph{#1}\mbox{}}
\fi

%%% Use protect on footnotes to avoid problems with footnotes in titles
\let\rmarkdownfootnote\footnote%
\def\footnote{\protect\rmarkdownfootnote}

%%% Change title format to be more compact
\usepackage{titling}

% Create subtitle command for use in maketitle
\newcommand{\subtitle}[1]{
  \posttitle{
    \begin{center}\large#1\end{center}
    }
}

\setlength{\droptitle}{-2em}
  \title{R Advanced Spatial Lessons}
  \pretitle{\vspace{\droptitle}\centering\huge}
  \posttitle{\par}
  \author{Ben Best}
  \preauthor{\centering\large\emph}
  \postauthor{\par}
  \predate{\centering\large\emph}
  \postdate{\par}
  \date{2017-09-24}

\usepackage{booktabs}
\usepackage{amsthm}
\makeatletter
\def\thm@space@setup{%
  \thm@preskip=8pt plus 2pt minus 4pt
  \thm@postskip=\thm@preskip
}
\makeatother

\usepackage{amsthm}
\newtheorem{theorem}{Theorem}[chapter]
\newtheorem{lemma}{Lemma}[chapter]
\theoremstyle{definition}
\newtheorem{definition}{Definition}[chapter]
\newtheorem{corollary}{Corollary}[chapter]
\newtheorem{proposition}{Proposition}[chapter]
\theoremstyle{definition}
\newtheorem{example}{Example}[chapter]
\theoremstyle{definition}
\newtheorem{exercise}{Exercise}[chapter]
\theoremstyle{remark}
\newtheorem*{remark}{Remark}
\newtheorem*{solution}{Solution}
\begin{document}
\maketitle

{
\setcounter{tocdepth}{1}
\tableofcontents
}
\chapter*{Prerequisites}\label{prereq}
\addcontentsline{toc}{chapter}{Prerequisites}

Lessons presented here are a continuation of the
\href{http://www.datacarpentry.org/lessons/\#geospatial-data-workshop}{Geospatial
workshop using R of Data Carpentry} described more specifically for the
\href{https://jsta.github.io/2017-09-27-LBNL/}{Lawrence Berkeley
National Lab: Sep 27-28, 2017}.

This content is setup for now using
\href{http://bookdown.org/yihui/bookdown}{bookdown} (using the
\href{https://github.com/rstudio/bookdown-demo}{bookdown-demo}) for
expediency, and meant to eventually be folded into the
\href{https://github.com/swcarpentry/styles}{Software Carpentry style}.

\chapter{Tidy Spatial Analysis}\label{tidy}

Resources:

\begin{itemize}
\tightlist
\item
  \href{http://strimas.com/r/tidy-sf/}{Tidy spatial data in R: using
  dplyr, tidyr, and ggplot2 with sf}
\end{itemize}

\section{Overview}\label{overview}

\textbf{Questions} - How to elegantly conduct complex spatial analysis?

\textbf{Objectives} - Understand how to use the ``then'' operator
\texttt{\%\textgreater{}\%} to pass output from one function into input
of the next. - Perform

\section{Things You'll Need to Complete this
Tutorial}\label{things-youll-need-to-complete-this-tutorial}

\textbf{R Skill Level}: Intermediate - you've got basics of R down.

You'll need \ldots{}

\section{Challenge: Explore Raster
Metadata}\label{challenge-explore-raster-metadata}

Without using the \texttt{raster} function to read the file into
\texttt{R}, determine the following about the
\texttt{NEON-DS-Airborne-Remote-Sensing/HARV/DSM/HARV\_DSMhill.tif}
file:

\begin{enumerate}
\def\labelenumi{\arabic{enumi}.}
\tightlist
\item
  Does this file has the same \texttt{CRS} as \texttt{DSM\_HARV}?
\item
  What is the \texttt{NoDataValue}?
\item
  What is resolution of the raster data?
\item
  How large would a 5x5 pixel area would be on the Earth's surface?
\item
  Is the file a multi- or single-band raster?
\end{enumerate}

Notice: this file is a \texttt{hillshade}. We will learn about
hillshades in Work with Multi-band Rasters: Images in R.

\subsection{Answers}\label{answers}

\begin{verbatim}
rows        1367 
columns     1697 
bands       1 
lower left origin.x        731453 
lower left origin.y        4712471 
res.x       1 
res.y       1 
ysign       -1 
oblique.x   0 
oblique.y   0 
driver      GTiff 
projection  +proj=utm +zone=18 +datum=WGS84 +units=m +no_defs 
file        data/NEON-DS-Airborne-Remote-Sensing/HARV/DSM/HARV_DSMhill.tif 
apparent band summary:
   GDType hasNoDataValue NoDataValue blockSize1 blockSize2
1 Float64           TRUE       -9999          1       1697
apparent band statistics:
        Bmin      Bmax     Bmean       Bsd
1 -0.7136298 0.9999997 0.3125525 0.4812939
Metadata:
AREA_OR_POINT=Area 
\end{verbatim}

\begin{enumerate}
\def\labelenumi{\arabic{enumi}.}
\tightlist
\item
  If this file has the same CRS as DSM\_HARV? Yes: UTM Zone 18, WGS84,
  meters.
\item
  What format \texttt{NoDataValues} take? -9999
\item
  The resolution of the raster data? 1x1
\item
  How large a 5x5 pixel area would be? 5mx5m How? We are given
  resolution of 1x1 and units in meters, therefore resolution of 5x5
  means 5x5m.
\item
  Is the file a multi- or single-band raster? Single.
\end{enumerate}

\section{Key Points}\label{key-points}

\begin{itemize}
\tightlist
\item
  The Coordinate Reference System or CRS tells R where the raster is
  located in geographic space and what method should be used to
  ``flatten'' or project the raster.
\end{itemize}

\chapter{Literature}\label{literature}

Here is a review of existing methods.

\chapter{Methods}\label{methods}

We describe our methods in this chapter.

\chapter{Applications}\label{applications}

Some \emph{significant} applications are demonstrated in this chapter.

\section{Example one}\label{example-one}

\section{Example two}\label{example-two}

\chapter{Final Words}\label{final-words}

We have finished a nice book.


\end{document}
